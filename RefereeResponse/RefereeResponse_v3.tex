\documentclass[11pt]{article}

\usepackage[usenames,dvips]{pstcol}

\usepackage{color}

\usepackage{amssymb}

\usepackage{amsmath}

\usepackage{graphicx,color}

%

%\input{symbols.tex}

%

\newcounter{question}

\newcommand{\noteq}{{\vskip0.1in\bf \large \color{red} [Q:{\color{black}\addtocounter{question}{1}\arabic{question}}]~}}

\newcommand{\noteQ}{{\bf \LARGE \color{blue} [Q: {\color{black}\addtocounter{question}{1}\arabic{question}}]~}}

\newcommand{\notea}{{\bf \large \color{red} [A]~}}

\newcommand{\noteA}{{\bf \LARGE \color{blue} [A]~}}

\newcommand{\noteqq}[1]{{\bf \large \color{red} [Q: {\color{black}#1}]~}}

\newcommand{\noteaa}[1]{{\bf \large \color{red} [A: {\color{black}#1}]~}}

%

\textwidth 6.5in

\textheight 9.4in

\topmargin -0.8in

\oddsidemargin 0.0in

\evensidemargin 0.0in

\parindent 0.0in

\def\baselinestretch{1.0}

\def \Russ {Translated from Russian. Fixed. Thanks.}

\def \F    {Fixed.}

\def \FT   {Fixed. Thanks.}

\def \D    {Done.}

\def \AP   {Basically Stylistic- Author's Prerogative.}



\begin{document}



\begin{flushright}

\today

\end{flushright}

%----------------------------------------------------------------



Dear JINST Editors,\\



\parindent 1.0in



We thank the referees for their very helpful and knowledgeable

comments. We have substantially rewritten the manuscript in response 

to the comments as detailed in the list below. 



There are several areas, however, in which the referees requested

substantially expanding the scope of the paper.  We envisioned this

paper as a proof of  the principle that the use of the fast timing

photodetectors presently being developed would enable the separation

of Cherenkov and scintillation light, as the reconstruction of the

direction of low energy ($\sim$1 MeV) electrons in a liquid

scintillator detector has never been shown before.  In this paper, we

want to: 1) show scintillation/Cherenkov separation is possible; 2)

determine the parameters of a large scintillator detector that are

important to this separation; and 3) demonstrate that reconstruction

techniques based on Cherenkov light would work in idealized

situations.



We are now working on two follow-up papers that are substantially

beyond the scope of the present paper, moving beyond

proof-of-principle into the laborious and time-intensive area of

reconstruction algorithms, areas for which the difficulty is known

well by some of us from our experience on KamLand. The first is a

detailed study of reconstruction in more realistic scenarios. The

second is a proposal for a $0\nu\beta\beta$ experiment  where the detailed

background reduction factors and sensitivity to new physics are

presented, SuperNEMO has a similar paper, but for scintillator

detectors nothing similar exists since a directional signal has not

been considered for them in the past.  Finally, we will be having a

turnover of young personnel so this paper provides an accessible summary

of their work.





\parindent 3.0in

                

    Sincerely,\\



\parindent 4.0in



		         Lindley Winslow, for the Authors



\parindent 0.0in



%\begin{verbatim}

%\end{verbatim}



\noteq [Referee 1 Intro] ``I like  very much the clean, transparent, simple

and well-motivated methodology (with one exception..). The structure

of the paper is nice''



\notea{\bf Many thanks.}\\





\noteq [Referee 1 Intro] ``However the presentation of the results is a

different thing... I am particularly concerned with the fact...,

instead of showing the resulting CosTheta distribution that would seem

like the obvious result one wants to see. .... Why not to disclose

this openly to allow the reader to evaluate the results in detail?''



\notea{\bf  This is our fault- the desired plot was (is) there

(now Figure 5), but was labeled as $p_x/|\vec{p}|$ rather than the obvious

$\cos\theta$ that it is. We have made this clear on the figure and in the

text. We have also added the distribution for the 2.1~MeV point. Thanks.}\\



\noteq [Referee 1: Page 1] ``Please specify FWHM or sigma and give a

reference for both cases.''

\notea{\bf  \D  ~We have added references to KamLAND and Borexino.}\\




\noteq [Referee 1: Page 1] ``...but please avoid misplacing the adjective good in

this context. ...'' 

\notea{\bf  \FT}\\



\noteq [Referee 1: Page 2] ``..the authors should..check and refer to

the work of NEXT and Super-Nemo collaborations:''



\notea{\bf  \D We have added a discussion of previous work on directionality.}\\





\noteq [Referee 1: Page 2] ``Define PPO here, not later''



\notea{\bf \D}\\





\noteq [Referee 1: Page 3] ``Please define all magnitudes in eq. 2.1 and 2.2''



\notea{\bf \D}\\ 





\noteq [Referee 1: Page 3]  ``Indicating the direction of the

electrons should not harm''



\notea{\bf  We were not sure what the referee was requesting. If this is a request for figure 1, we have tried adding the arrows and it becomes too busy. We think the caption description works well.}\\






\noteq [Referee 1: Page 3] ``None of the best limits or claims in bb0

uses an isotope that satisfies this condition..''



\notea{\bf \FT We have rewritten this section to address this comment.}\\


\noteq [Referee 1: Page 4] ``could you perhaps include the

attenuation @400nm and n@400..?''



\notea{\bf \D We have added the attenuation length at 400nm. We have also added the n at other wavelengths in other locations. Unfortunately we cannot cite one figure in the references as the index of refraction is taken from the KamLAND simulation which is based on several measurements that are in the cited references.}\\



\noteq [Referee 1: Page 4] ``Could you please include also the cutoff

wavelength previously referred to?''


\notea{\bf This has been added.}\\


\noteq [Referee 1: Page 4] `` Scattering have not been included..'' 



\notea{\bf We have added a figure and accompanying text to show that

  the scattering only adds a late tail. Thanks.}\\



\noteq [Referee 1: Page 4] ``For which digitized values come from ??''



\notea{\bf \F}\\



\noteq [Referee 1: Page 4] ``5 MeV electrons.. It is very disturbing

to benchmark the simulation with electrons having an energy doubling

the Qbb of the most common isotopes..'' 



\notea{\bf We have clarified in several places 

that the 5 MeV case is for neutrino-electron scattering experiments. For $0\nu\beta\beta$ we

have made it explicit that we consider 1.4 and 2.1 MeV}\\



\noteq [Referee 1: Page 4] ``Please specify also the track length in

the direction of emission''



\notea{\bf \D}\\



\noteq [Referee 1: Page 5] ``Before the solutes in liquid... Please rephrase'' 



\notea{\bf \D}\\



\noteq [Referee 1: Page 5] ``I hope that [43] is not the final version

but a draft. 



\notea{\bf It was a draft; we have changed the reference. Thanks!}\\





\noteq [Referee 1: Page 6] ``So I would be very much convinced ..if

the authors would report Fig. 3 during the electron history

(equivalently, energy or distance, including several histograms for

each case. It would be nice to see at which energy scale or length the

electron isotropizes and the Cherenkov cone smears out.''



\notea{\bf  In the updated figure 4 (angular PE hit distribution after time cut), we added the histograms for 1.4~MeV and 2.1~MeV. This shows that also for 2.1 and 1.4~MeV the PE hits are not isotropized, including the full history of the electron tracks. We have also added the path lengths and total distance and time from the vertex to the text. We also present in Fig.~\ref{truthscatt} of this report  more raw data for the referees which supports the statement that the scattering angle is small while the majority of the Cherenkov light is produced.}\\ 

\begin{figure}
        \begin{center}
        \includegraphics[scale=0.50]{Figure1_1.pdf} 
        \includegraphics[scale=0.50]{Figure1_2.pdf} 
         \caption[]{"For electrons with and initial energy of 1.4~MeV, we show (Top) the angle between the initial electron direction and the direction at 0.7 MeV (=half of the initial electron energy) and (Bottom) the angle between the initial electron direction and the direction of the last Geant4 step before the electron stops. We note that due to scattering the final direction of the electron before it stops does not correspond to the initial direction; however the scattering angle is small while the majority of Cherenkov light is produced.}
        \end{center}
\end{figure}



\noteq [Referee 1: Pages 7-8] ``This is interesting, but specially

section 6 seems a bit off-topic to me. I am not against keeping it...''



\notea{\bf We'd like to keep it. }\\





\noteq [Referee 1: Page 9] ``Please add a histogram of the reconstructed

angle for all energies...'' 



\notea{\bf  \D}\\




\noteq [Referee 1: Page 9] ``..the vertex reconstruction has to be

obtained from all photons and the direction from the Cherenkov photons..''



\notea{\bf We agree with the referee and plan to address this in future papers. To do a credible and useful job on the position dependence will require a fairly-detailed vertex reconstruction. This is beyond the scope of this present �proof-of-principle� paper. We have tried to clarify this in the text }\\





\noteq [Referee 1: Fig. 7] ``the authors should show the spread of the

cos(theta) distribution for each of the energies. 



\notea{\bf \D See Figure 5.}\\




\noteq [Referee 1: Fig. 7 ``It would be nice to show/indicate..also the

Cherenkov threshold for high wave-length (n about 1.4)



\notea{\bf We have added this to the text.}\\



\noteq [Referee 1: Conclusions] ``Should be rewritten after the

previous objections are addressed.'' 



\notea{\bf \D Thanks!}\\




\noteq [Referee 1: Page 7-8] ``I see no convincing proof of it with

the information here presented. Hope the answers help clarifying the issue.'' 



\notea{\bf We have made the sentence more specific. Thanks.}\\



%\newpage

\noteq [Referee 2: General Comments] ``The paper introduces an

original technical development which could increase the interest of

large scale scintillation detectors for neutrino physics in general

and neutrinoless double beta decay in particular''



\notea{\bf  Thank you, this warms our hearts.}\\



\noteq [Referee 2: General comments] `` ..however the discussion

should deserve...a deeper analysis.''



\notea{\bf Yes. We included more data from the lower energy studies,

  have extracted much more detail from the simulation, and have

  extensively rewritten the manuscript. It is still at the

  `proof-of-principle level. Our next work will include a full sensitivity analysis including \a deeper understanding of the performance of the reconstruction algorithms with a more detailed simulation. This is a big job and we are working on it! }\\



\noteq [Referee 2: General Comments] ``The physics case is included in

the title and mentioned in the text but never discussed in some

detail...A description of why the method is powerful for double beta

decay, and..to what extent it can improve the experimental sensitivity

would complete the discussion



\notea{\bf  \D}\\


\noteq [Referee 2: General Comments] ``The chosen example (5 MeV,

head-to-head single electrons) is of little interest for double beta decay.'' 



\notea{\bf We have added plots and numbers for 1.4 and 2.1 MeV, and

  substantially rewritten the text to differentiate between the cases

  of neutrino-electron scattering and double-beta decay.}\\





\noteq [Referee 2: General Comments] ``The dependence on the decay

position.. is not discussed.'' 



\notea{\bf To do a credible and useful job on the position dependence

  will require a fairly-detailed vertex reconstruction. This is beyond

  the scope of this present `proof-of-principle' paper. We have tried

  to clarify this in the text. }\\





\noteq [Referee 2: General Comments] Threshold effects as well as the

effect of statistical fluctuations for the low energy tails.. are not

discussed. The same holds for position and direction reconstruction

capabilities'' 



\notea{\bf Statistical fluctuations are addressed by the studies at lower energies and we mention explicitly in the conclusions the widening of the distributions due to decreased statistics. Threshold effect are outside of the scope of this paper. In general in these detectors, the largest loss of photon statistics is a the photocathode. The electronics for PMT readout are very efficient. }\\





\noteq [Referee 2: Abstract] ``While appropriate for neutrino physics,

energy resolution is definitely worse with respect to other techniques..''



\notea{\bf We have fixed the text. Thanks.}\\





\noteq [Referee 2: Introduction] ``A reference ... would help to understand

the connection with Majorana neutrinos.''



\notea{\bf \D}\\





\noteq [Referee 2: Page 1] Is this 1 sigma or FWHM? 



\notea{\bf It's sigma- we have fixed the text.}\\





\noteq [Referee 2: Page 2] ``better than water Cherenkov detectors'' 



\notea{\bf \FT}\\


\noteq [Referee 2: Page 2 ] ``strong suppressant of backgrounds''..this

statement should deserve a deeper quantitative discussion ..''



\notea{\bf We agree. However, a useful ``quantitative discussion''

  will require a vertex reconstruction. We will get there, but at present

  it's beyond the scope of `proof-of-principle, looking to see if we

  can see the signal. We have tried to make this clearer in the text. }\\



\noteq [Referee 2: Equ. 2.1] ``..the meaning of the parameter `x' is

not described.''



\notea{\bf \F}\\



\noteq [Referee 2: Page 3] What is the typical cutoff value and the

corresponding useful fraction of Cherenkov light? 



\notea{\bf We have added the cutoff values.}\\




\noteq [Referee 2: Page 3] `` `Probable case being...' Events with

asymmetric electron energies are far more abundant.'' 



\notea{\bf \FT}\\




\noteq [Referee 2: Page 4] `` `Fully absorbing' ''. How are results

influenced by deviations from this hypothesis?''



\notea{\bf This is a level of detail well beyond what we've done so

  far; once we have a real detector design and a full simulation it

  can be addressed, but at this point devoting effort to it 

doesn't seem productive.}\\



\noteq [Referee 2: Page 4] ``Single 5 MeV electrons''.  



\notea{\bf \F}\\





\noteq [Referee 2: Page 4] ``I believe the `realistic' analysis

mentioned here is indispensable ..for this process'' 



\notea{\bf We agree, and are working on it. It isn't easy, or,

  unfortunately, fast compared to a post-doc tenure.}\\





\noteq [Referee 2: Page 6] ``The dependence on the position and the

corresponding transit time variations cannot in my opinion be neglected.'' 



\notea{\bf This part of the paper describes a simplified test simulation (where the photodetector TTS was switched off) to extract the time scales of various effects. We made changes in the text to make this clear.}\\





\noteq [Referee 2: Figure 3] `` This figure is very interesting

to understand the capacity of the method to maintain directionality

information... The comparison with a single event (or equivalently the

distribution of the variances) and the distribution for 1 MeV

electrons would improve to understand also the limits/power of the method'' 



\notea{\bf  We have added to what is now figure 4 the distributions for the two lower energies. We think this is a nice illustration of the limits/power of the method.}\\





\noteq [Referee 2: reconstruction] ``It is not clear if limiting the

reconstruction to Cherenkov photons improve reconstruction'' 



\notea{\bf Yes, we agree. In this paper we were focussing on the Cherenkov photons as a simple way to apply direction reconstruction algorithms. In the future we will work to reconstruct position and direction simultaneously from the scintillation and Cherenkov light.}\\

\end{document}



\noteq [Referee 1] 

\notea{\bf }\\



